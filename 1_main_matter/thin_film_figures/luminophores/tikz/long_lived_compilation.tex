\documentclass[tikz, usenames, dvipsnames]{standalone}
\usepackage{pgfplots}
\usepgfplotslibrary{groupplots}
\pgfplotsset{width=12.cm, height=7cm, compat=1.18}
\usepackage{tikz}
\usepackage{gensymb}
\usepackage{amsmath}
\usepgfplotslibrary{fillbetween}

\definecolor{darkgray176}{RGB}{176,176,176}
\DeclareMathSymbol{\cdot}{\mathord}{symbols}{"01} % shortens space around cdot

\begin{document}
	
	\begin{tikzpicture}
		\begin{groupplot}[
			group style={
				group size= 1 by 2,
				group name=abs_group,
				x descriptions at=edge bottom,
				y descriptions at=edge left,
			},
			xmin=210, xmax=800,
			]
			
			\nextgroupplot[
			tick align=outside,
			tick pos=left,
			height=5cm,
			x grid style={darkgray176},
			xmin=210, xmax=700,
			xtick style={color=black},
			y grid style={darkgray176},
			scaled y ticks=false,
			ytick={5e4, 1E5, 1.5E5, 2e5, 2.5E5},
			yticklabels={5,10,15,20,25},
			ymin=3e4, ymax=2.7e5,
			ytick style={color=black},
			]
			
			% PtOEP
			\addplot[draw=Maroon, very thick, smooth, line join=round] table[col sep=comma, skip first n=1]{CSVs/long_lived/PtOEP_abs.csv};
			\addplot[draw=Maroon, very thick, smooth, line join=round, densely dashed] table[col sep=comma, skip first n=1]{CSVs/long_lived/PtOEP_em.csv};
			
			% Pyrene
			\addplot[draw=blue, very thick, smooth, line join=round] table[col sep=comma, skip first n=1]{CSVs/long_lived/Pyrene_abs.csv};
			\addplot[draw=blue, very thick, smooth, line join=round, densely dashed] table[col sep=comma, skip first n=1]{CSVs/long_lived/Pyrene_em.csv};
			
			% Ru-bpy
			\addplot[draw=Rhodamine, very thick, smooth, line join=round] table[col sep=comma, skip first n=1]{CSVs/long_lived/Ru-bpy_abs.csv};
			\addplot[draw=Rhodamine, very thick, smooth, line join=round, densely dashed] table[col sep=comma, skip first n=1]{CSVs/long_lived/Ru-bpy_em.csv};
			
			\nextgroupplot[
			legend cell align={left},
			legend style={fill opacity=1, draw opacity=1, text opacity=1, draw=black, anchor=south,xshift=-0.0pt, yshift=+0.0pt, cells={align=left}, legend columns=2, transpose legend},
			tick align=outside,
			tick pos=left,
			x grid style={darkgray176},
			xlabel={Wavelength (nm)},
			xmin=210, xmax=700,
			xtick style={color=black},
			y grid style={darkgray176},
			scaled y ticks=false,
			ytick={0, 1E4, 2E4, 3E4},
			yticklabels={0,1,2,3},
			ymin=0., ymax=30000.,
			ytick style={color=black},
			yshift=1cm,
			legend to name=mylegend,
			]
			\addplot[draw=black, solid, very thick] coordinates {
				(600, -1e-3)
				(700, -1e-3)};
			\addlegendentry{Absorption}
			\addplot[draw=black, densely dashed, very thick] coordinates {
				(600, -1e-3)
				(700, -1e-3)};
			\addlegendentry{Emission}
			\addplot[draw=Green, fill=Green, area legend] (1,1);
			\addlegendentry{HPTS}
			\addplot[draw=Maroon, fill=Maroon, area legend] (1,1);
			\addlegendentry{PtOEP}
			\addplot[draw=blue, fill=blue, area legend] (1,1);
			\addlegendentry{Pyrene}
			\addplot[draw=Rhodamine, fill=Rhodamine, area legend] (1,1);
			\addlegendentry{Ru-bpy}
			\addplot[draw=red, fill=red, area legend] (1,1);
			\addlegendentry{Ru-dpp}
			\addplot[draw=orange, fill=orange, area legend] (1,1);
			\addlegendentry{Ru-pn}
			
			\addplot[draw=none, name path=xabs] coordinates {
				(210, -1e-3)
				(600, -1e-3)};
			\addplot[draw=none, name path=xem] coordinates {
				(450, -1e-3)
				(650, -1e-3)};
			
			% HPTS
			\addplot[draw=Green, very thick, smooth, line join=round, name path=hptsabs] table[col sep=comma, skip first n=1]{CSVs/photochemcad/hpts_abs.csv};
			\addplot[draw=Green, very thick, smooth, line join=round, densely dashed, name path=hptsem] table[col sep=comma, skip first n=1]{CSVs/photochemcad/hpts_em.csv};
			\addplot[Green!30] fill between[of=hptsabs and xabs];
			\addplot[Green!30] fill between[of=hptsem and xem];
			
			% PtOEP
			\addplot[draw=Maroon, very thick, smooth, line join=round] table[col sep=comma, skip first n=1]{CSVs/long_lived/PtOEP_abs.csv};
			\addplot[draw=Maroon, very thick, smooth, line join=round, densely dashed] table[col sep=comma, skip first n=1]{CSVs/long_lived/PtOEP_em.csv};
			
			% Pyrene
			\addplot[draw=blue, very thick, smooth, line join=round] table[col sep=comma, skip first n=1]{CSVs/long_lived/Pyrene_abs.csv};
			\addplot[draw=blue, very thick, smooth, line join=round, densely dashed] table[col sep=comma, skip first n=1]{CSVs/long_lived/Pyrene_em.csv};
			
			% Ru-bpy
			\addplot[draw=Rhodamine, very thick, smooth, line join=round] table[col sep=comma, skip first n=1]{CSVs/long_lived/Ru-bpy_abs.csv};
			\addplot[draw=Rhodamine, very thick, smooth, line join=round, densely dashed] table[col sep=comma, skip first n=1]{CSVs/long_lived/Ru-bpy_em.csv};
			
			% Ru-dpp
			\addplot[draw=red, very thick, smooth, line join=round] table[col sep=comma, skip first n=1]{CSVs/long_lived/Ru-dpp_abs.csv};
			\addplot[draw=red, very thick, smooth, line join=round, densely dashed] table[col sep=comma, skip first n=1]{CSVs/long_lived/Ru-dpp_em.csv};
			
			% Ru-pn
			\addplot[draw=orange, very thick, smooth, line join=round] table[col sep=comma, skip first n=1]{CSVs/long_lived/Ru-pn_abs.csv};
			\addplot[draw=orange, very thick, smooth, line join=round, densely dashed] table[col sep=comma, skip first n=1]{CSVs/long_lived/Ru-pn_em.csv};
			
		\end{groupplot}
		
		\node [right=-13.5mm,anchor=north, rotate=90] at
		($(abs_group c1r1.north west)!0.5!(abs_group c1r2.south west)$)
		{Absorption $\varepsilon$ ($10^4$.cm$^{-1}$.M$^{-1}$)};
		
		
		%%%%%%%%%%%%%%%
		% Emission plot
		%%%%%%%%%%%%%%%
		
		\begin{groupplot}[
			group style={
				group size= 1 by 2,
				group name=em_group,
				x descriptions at=edge bottom,
				y descriptions at=edge left,
			},
			xmin=210, xmax=800,
			]
			
			\nextgroupplot[
			height=5cm,
			axis y line*=right,
			axis x line=none,
			tick align=outside,
			xmin=210, xmax=700,
			scaled y ticks=false,
			ytick={5e4, 10E4, 15E4, 20E4, 25E4},
			yticklabels={5,10,15,20,25},
			ymin=3e4, ymax=2.7e5,
			ytick style={color=black},
			]
			
			\addplot coordinates{
				(600,-1e-3)
				(600,-1e-3)
			};
			
			\nextgroupplot[
			axis y line*=right,
			axis x line=none,
			tick align=outside,
			xmin=210, xmax=700,
			scaled y ticks=false,
			ytick={0, 1E4, 2E4, 3E4},
			yticklabels={0,1,2,3},
			ymin=0., ymax=3e4,
			ytick style={color=black},
			yshift=1cm,
			]
			\addplot[draw=black, solid, very thick] coordinates {
				(600, -1e-3)
				(700, -1e-3)};
			
		\end{groupplot}
		
		\node [right=+7.5mm,anchor=north, rotate=90] at
		($(em_group c1r1.north east)!0.5!(abs_group c1r2.south east)$)
		{Emission $\varepsilon \cdot \Phi_\mathrm{F}$ (A.U.)};
		
		\node [right=0mm,anchor=south] at
		($(abs_group c1r1.north west)!0.5!(abs_group c1r1.north east)$)
		{\pgfplotslegendfromname{mylegend}};
		
		\pgfresetboundingbox
		\path(-1.3,4.6) rectangle + (13.,-11.15);
		% Fo unkwnown reasons, the "fill between" command creates some whitespace below, and "clip" does not work either. There is surely a very good reason to this state of affair, but I can't help banging my head on my keyboard, cursing the devs who made this happen.
		%\clip (-1.3,4.6) rectangle + (13.,-11.15);
	\end{tikzpicture}
\end{document}