\documentclass[tikz]{standalone}
\usepackage{pgfplots}
\usepgfplotslibrary{groupplots}
\pgfplotsset{width=10cm, height=7cm, compat=1.18}
\usepackage{tikz}
\usepackage{gensymb}
\usepackage{amsmath}

\begin{document}
	
	\begin{tikzpicture}
		
		
		\begin{axis}[
			tick align=outside,
			tick pos=left,
			x grid style={black},
			xlabel={pO$_2$ (mmHg)},
			xmin=0, xmax=150,
			xtick style={color=black},
			y grid style={black},
			ylabel={oxygen saturation (\%)},
			ymin=0.0, ymax=105,
			ytick style={color=black},
			]
			\addplot [draw=red, thick, smooth]
			table[col sep=semicolon]{CSVs/severinghaus1966.csv};
			% NOTE about the code below :
			% Actually, we can't really do this because this curve is not the same for the arterial and veinous blood, due to pH changes caused by more CO2. See Nunn's Respiratory Physiology chapter 7.
			%\draw[red, radius=1.2, fill=red, anchor=center] (axis cs:100,97.5) circle node[below]{$A$}; % Arterial
			%\draw[red, radius=1.2, fill=red, anchor=center] (axis cs:40,74.3) circle node[left]{$V$}; % Veinous
			%
		\end{axis}
		\begin{axis}[
			axis x line* = top,
			axis y line = none,
			xtick align=outside,
			xmin=0, xmax= 150/760*100,
			xlabel={pO$_2$ (kPa)}]
			\addplot[opacity=0,domain=2.5:7.5] {x};
		\end{axis}
	\end{tikzpicture}
	
\end{document}