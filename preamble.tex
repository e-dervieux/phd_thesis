% Global packages
\documentclass[a4paper,table]{book}
\overfullrule=5pt % This draws black rectangle at overfullbox
\usepackage[a4paper]{geometry}
\newgeometry{top=20mm, bottom=20mm, marginparwidth=80pt}
\usepackage[english]{babel}
% see, https://wayback-api.archive.org/web/20240521033957/https://en.wikibooks.org/wiki/LaTeX/Text_Formatting#Hyphenation
% for more information on word hyphenation
\hyphenation{long la-tex}
\hyphenation{tri-sodium}
\hyphenation{haemo-globin}
\hyphenation{hemo-globin}
\hyphenation{con-cen-tra-tion}
\hyphenation{ery-thro-cyte}
\hyphenation{ery-thro-cytes}
\hyphenation{ery-thro-cy-tic}
\hyphenation{meta-bisulfite}

% stretching / hyphenation in texttt
% https://tex.stackexchange.com/questions/171278/
\setlength{\emergencystretch}{2em}
\newcommand{\origttfamily}{}%
\let\origttfamily=\ttfamily%
\renewcommand{\ttfamily}{\origttfamily \hyphenchar\font=`\-}
\renewcommand*{\thefootnote}{(\arabic{footnote})}
\usepackage{eurosym}

%%%%%%%%%%%%
% Math stuff
%%%%%%%%%%%%
\usepackage{amsmath}
\newcommand{\spminus}{^{\smash{-}}} % smashed minus
\DeclareMathOperator{\cotan}{cotan}

% Closed square root
\usepackage{letltxmacro}
\makeatletter
\let\oldr@@t\r@@t
\def\r@@t#1#2{%
	\setbox0=\hbox{$\oldr@@t#1{#2\,}$}\dimen0=\ht0
	\advance\dimen0-0.2\ht0
	\setbox2=\hbox{\vrule height\ht0 depth -\dimen0}%
	{\box0\lower0.4pt\box2}}
\LetLtxMacro{\oldsqrt}{\sqrt}
\renewcommand*{\sqrt}[2][\ ]{\oldsqrt[#1]{#2} }
\makeatother

\usepackage{esint}
\usepackage{cancel}
\usepackage{bbold}
\DeclareMathOperator\erf{erf}
\DeclareMathOperator\erfi{erfi}
\DeclareMathOperator\erfc{erfc}
\DeclareMathOperator*{\argmax}{arg\,max}
\DeclareMathOperator*{\argmin}{arg\,min}
\newcommand{\var}{\mathbf{Var}}
\newcommand{\esp}{\mathbf{E}}
\newcommand{\cov}{\mathbf{Cov}}
\newcommand{\ind}{\mathrel{\perp\!\!\!\perp}}
\newcommand{\e}{\mathrm{e}}
\usepackage{bm}

\makeatletter
\newcommand{\pushright}[1]{\ifmeasuring@#1\else\omit\hfill$\displaystyle#1$\fi\ignorespaces}
\newcommand{\pushleft}[1]{\ifmeasuring@#1\else\omit$\displaystyle#1$\hfill\fi\ignorespaces}
\makeatother

\usepackage{afterpage}
\usepackage{pdflscape}
%\newenvironment{rotatepage}%
%{\clearpage\pagebreak[4]\global\pdfpageattr\expandafter{\the\pdfpageattr/Rotate 90}}%
%{\clearpage\pagebreak[4]\global\pdfpageattr\expandafter{\the\pdfpageattr/Rotate 0}}%

%%%%%%%%%%%%%%%%%%%%%
% Table related stuff
%%%%%%%%%%%%%%%%%%%%%
\usepackage{multirow}
\usepackage{multicol}
\usepackage[figuresright]{rotating}
\usepackage{tabularx}
\newcommand{\specialcell}[2][c]{%
	\begin{tabular}[#1]{@{}c@{}}#2\end{tabular}}

% Allows math to be bolded by textbf
% see: https://tex.stackexchange.com/questions/347146/using-textbf-to-make-also-math-bold-in-sentence
\AddToHook{cmd/bfseries/after}{\boldmath}
\AddToHook{cmd/normalfont/after}{\unboldmath}
% \AddToHook{cmd/itshape/after}{XXX} <- in case it's needed

% Allow aligning with respect to the middle of a symbol, see here:
% https://tex.stackexchange.com/questions/499883/align-by-center-of-symbol
\newcommand{\crel}[1]{%
	\global\setbox1=\hbox{$#1$}%
	\global\dimen1=0.5\wd1
	\mathrel{\hbox to\dimen1{$#1$\hss}}&\mathrel{\mspace{-\thickmuskip}\hbox to\dimen1{}}%
}

% TODO: see later about this for numbering depth adjustments
% from : https://tex.stackexchange.com/questions/300821/remove-chapter-numbers-keeps-section-numbers
%\setcounter{secnumdepth}{4} % <- used on a per-chapter basis
%\usepackage{chngcntr}
%\counterwithout{section}{chapter}

\makeatletter
\newcommand*{\centerfloat}{%
	\parindent \z@
	\leftskip \z@ \@plus 1fil \@minus \textwidth
	\rightskip\leftskip
	\parfillskip \z@skip}
\makeatother

% For vertical spacing in table
\newcommand\Tstrut{\rule{0pt}{2.6ex}}         % = `top' strut
\newcommand\Bstrut{\rule[-1.4ex]{0pt}{0pt}}   % = `bottom' strut

% - END OF TABLE STUFF -

% Graphic related stuff
\usepackage{graphicx}
\usepackage[skins]{tcolorbox}
\usepackage{luacolor}
\usepackage[soul]{lua-ul}
\usepackage{pdfpages}
\usepackage{tikz}
\usepackage{pgf-pie} %TODO: remove this?
\usepackage{xcolor}
\definecolor{lightgreen}{RGB}{200, 255, 200}
\definecolor{minotaur}{RGB}{225, 6, 0}
\definecolor{house}{RGB}{35, 91, 168}
\newcommand{\myblood}{blood}
\usepackage{placeins} % bring FloatBarrier
\usepackage{hhline}
\usepackage[export]{adjustbox} % brings valign
\usepackage{layouts} % for debug and size information
\usepackage{bigdelim}

% Notes, and todos
\usepackage{csquotes}
\usepackage{sidenotes}
\usepackage{todonotes}

% Symbols
\usepackage{stmaryrd} % Where \llbracket and \rrbracket are defined
\usepackage{enumitem, amssymb}
\usepackage{textcomp, gensymb}
\usepackage{pifont}
\usepackage{circledsteps}
% \usepackage[nointegrals]{wasysym} <- causes several warnings
\def\female{\mbox{\fontencoding{U}\fontfamily{wasy}\selectfont\char25}}
\def\male{\mbox{\fontencoding{U}\fontfamily{wasy}\selectfont\char26}}

% Chemistry packages
\usepackage[version=4]{mhchem} % <- for chemical equations

\usepackage{chemfig} % <- for molecules drawings
\setchemfig{atom sep=2em}

\usepackage{chemarr}
\usepackage{chemmacros}
\chemsetup[reactions]{own-counter=false}

% Bibliography
\usepackage[backend=biber, bibencoding=utf8, sorting=none, style=numeric-comp, backref=true]{biblatex}
\addbibresource{clean_bibliography.bib}

% hyperref: load this at last !
% hypperef doc https://mirrors.ircam.fr/pub/CTAN/macros/latex/contrib/hyperref/doc/hyperref-doc.pdf
\usepackage[hyperfootnotes=false]{hyperref}
\hypersetup{%
	pdfauthor={Emmanuel DERVIEUX},%
	pdftitle={Dervieux PhD Thesis},%
	bookmarksnumbered=true,%
	colorlinks=false,%
	citecolor=black,%
	filecolor=black,%
	linkcolor=black,%
	urlcolor=black,%
	pdfstartview=FitH,%
}

% Glossary
\usepackage[toc, shortcuts]{glossaries}
% Acronyms: label, short name, full name

% From there: https://tex.stackexchange.com/questions/79907/howto-achieve-capitalized-description-in-glossary-table
% EDIT: https://tex.stackexchange.com/questions/717622/
\let\firstchar\uppercase
\let\oldprintglossaries\printglossaries
\def\printglossaries{\oldprintglossaries\let\firstchar\lowercase}

% Others
\newacronym{who}{WHO}{World Health Organisation}
\newacronym{fda}{FDA}{Food and Drug Administration}
\newacronym{fdm}{FDM}{Fused Deposition Modeling}
\newacronym{pas_fr}{PAS}{Plan d'Analyse Statistique}
\newacronym{tldr}{TL;DR}{Too Long; Didn't Read}

% Gases
\newacronym{co2}{CO$_2$}{\firstchar{C}arbon dioxide}
\newacronym{co}{CO}{\firstchar{C}arbon monoxide}
\newacronym{o2}{O$_2$}{\firstchar{D}i-oxygen}
\newacronym{n2}{N$_2$}{\firstchar{D}i-nitrogen}
\newacronym{voc}{VOC}{Volatile Organic Compound}

% Medecine
\newacronym{o2sat}{saO$_2$}{arterial \firstchar{D}i-oxygen saturation}
\newacronym{spo2}{spO$_2$}{\firstchar{P}ulsed di-oxygen saturation}
\newacronym{co2sat}{saCO$_2$}{arterial \firstchar{C}arbon dioxide saturation}
\newacronym{capco2}{cap-pCO$_2$}{\firstchar{C}apillary CO$_2$ pressure}
\newacronym{cosat}{SatCO}{\firstchar{C}arbon monoxide saturation}
\newacronym{pi}{PI}{Perfusion Index}
\newacronym{punit}{P.U.}{Perfusion Units}
\newacronym{copd}{COPD}{Chronic Obstructive Pulmonary Disease}
\newacronym{ldf}{LDF}{Laser Doppler Flowmetry}
\newacronym{cvc}{CVC}{Cutaneous Vascular Conductance}
\newacronym{map}{MAP}{Mean Arterial Pressure}
\newacronym{tewl}{TEWL}{Transepidermal Water Loss}
\newacronym{bmi}{BMI}{Body Mass Index}

% Hb species
\newacronym{sulfhb}{SulfHb}{Sulf-Haemo\-globin}
\newacronym{cnhb}{HiCN}{Cyan-Met-Haemo\-globin}
\newacronym{methb}{MetHb}{Met-Haemo\-globin}
\newacronym{o2hb}{O$_2$Hb}{Oxy-Haemo\-globin}
\newacronym{cohb}{COHb}{Carboxy-Haemo\-globin}
\newacronym{co2hb}{CO$_2$Hb}{Carbamino-Haemo\-globin}
\newacronym{hb}{HHb}{Deoxy-Haemo\-globin}

% Electronic
\newacronym{led}{LED}{Light Emitting Diode}
\newacronym{mems}{MEMS}{Micro Electro-Mechanical Systems}
\newacronym{isfet}{ISFET}{Ion-Selective Field-Effect Transistor}
\newacronym{fet}{FET}{Field-Effect Transistor}
\newacronym{mosfet}{MOSFET}{Metal Oxyde Semiconductor Field-Effect Transistor}
\newacronym{ntc}{NTC}{Negative Temperature Coefficient}
\newacronym{pcb}{PCB}{Printed Circuit Board}
\newacronym{smd}{SMD}{Surface Mount Device}
\newacronym{lfo}{LFO}{Low Frequency Oscillator}
\newacronym{bnc}{BNC}{Bayonet Neill–Concelman}
\newacronym{sma}{SMA}{SubMiniature version A}
\newacronym{psu}{PSU}{Power Supply Unit}
\newglossaryentry{ac}{name=AC,description={Alternative Current (by extension, anything time varying)}}
\newglossaryentry{dc}{name=DC,description={Direct Current (by extension, anything constant)}}
\newacronym{fwhm}{FWHM}{Full Width at Half Maximum}
\newacronym{adc}{ADC}{Analog to Digital Converter}
\newacronym{tia}{TIA}{TransImpedance Amplifier}
\newacronym{dac}{DAC}{Digital to Analog Converter}
\newacronym{dspcpu}{DSP}{Digital Signal Processor}

% Computer Science
\newacronym{uart}{UART}{Universal Asynchronous Receiver Transmitter}
\newacronym{usb}{USB}{Universal Serial Bus}
\newacronym{csv}{CSV}{Comma-Separated Values}
\newacronym{api}{API}{Application Programming Interface}

%%% Chemicals %%%

% Chemicals - Dyes
% Chemicals - Dyes - Fluorescent
\newacronym{hpts}{HPTS}{1-hydroxy-pyrene-3,6,8-trisulfonate} % a.k.a.:
\newacronym{hopsa}{HOPSA}{\firstchar{T}risodium salt of 8-hydroxyl-1,3,6-pyridine trisulfonic acid} % a.k.a.:
\newacronym{hpta}{HPTA}{\firstchar{H}ydroxypyrenetrisulfonic acid} % a.k.a. pyranine
\newacronym{rudpp}{Ru-dpp}{\firstchar{T}ris(4,7-diphenyl-1,10 phenanthroline) ruthenium (II) dichloride}
\newacronym{rupn}{Ru-pn}{\firstchar{T}ris(1,10 phenanthroline) ruthenium (II) dichloride}
\newacronym{rubpy}{Ru-bpy}{\firstchar{T}ris(2,2'-bipyridyl) ruthenium (II) dichloride}
% Chemicals - Dyes - Absorption
\newacronym{btb}{BTB}{BromoThymol Blue}
\newacronym{tb}{TB}{Thymol Blue}
\newacronym{mcp}{m-Cresol Purple}{MCP}
\newacronym{naf}{NAF}{$\alpha$-naphtolphtalein}
\newacronym{nbb}{NBB}{Naphtol Blue Black}

% Chemicals - Polymers
\newacronym{evoh}{EVOH}{\firstchar{E}thylene-vinyl alcohol}
\newacronym{hpmc}{HPMC}{\firstchar{H}ydroxy propyl methylcellulose}
\newacronym{pan}{PAN}{\firstchar{P}olyacrylonitrile}
\newacronym{pdms}{PDMS}{\firstchar{P}olydimethylsiloxane}
\newacronym{peg}{PEG}{\firstchar{P}olyethylene glycol}
\newacronym{pen}{PEN}{\firstchar{P}olyethylene naphthalate}
\newacronym{pet}{PET}{\firstchar{P}olyethylene terephthalate}
\newacronym{pibm}{polyIBM}{\firstchar{P}oly(isobutyl methacrylate)}
\newacronym{pla}{PLA}{\firstchar{P}oly-lactic acid}
\newacronym{pmma}{PMMA}{\firstchar{P}oly(methyl methacrylate)}
\newacronym{ppta}{PPTA}{\firstchar{P}oly(p-phenylene terephthalamide)}
\newacronym{ptfe}{PTFE}{\firstchar{P}olytetrafluoroethylene}
\newacronym{pva}{PVA}{\firstchar{P}olyvinyl alcohol}
\newacronym{pvc}{PVC}{\firstchar{P}olyvinyl chloride}
\newacronym{pvcd}{PVCD}{\firstchar{P}oly(vinylidene chloride-co-vinyl chloride)}
\newacronym{pvdf}{PVDF}{\firstchar{P}olyvinylidene fluoride}

% Chemicals - Phase Transfer Agents
\newacronym{teah}{TEAH}{\firstchar{T}etraethyl ammonium hydroxide}
\newacronym{toah}{TOAH}{\firstchar{T}etraoctyl ammonium hydroxide}
\newacronym{toab}{TOAB}{\firstchar{T}etraoctyl ammonium bromide}
\newacronym{tmah}{TMAH}{\firstchar{T}etramethyl ammonium hydroxide}
\newacronym{ctah}{CTAH}{\firstchar{C}etyltrimethyl ammonium hydroxide}

% Chemicals - Solvents
\newacronym{dmf}{DMF}{\firstchar{D}imethylformamide}
\newacronym{dmso}{DMSO}{\firstchar{D}imethyl sulfoxide}
\newacronym{ipa}{IPA}{\firstchar{I}sopropyl alcohol}
\newacronym{thf}{THF}{\firstchar{T}etrahydrofuran}

% Chemicals - Others
\newacronym{edta}{EDTA}{Ethylene Diamine Tetraacetic Acid}
\newacronym{nadh}{NADH}{Oxydised Nicotinamide Adenine Dinucleotide}
\newacronym{rh}{RH}{Relative Humidity}
\newacronym{rtil}{RTIL}{Room Temperature Ionic Liquid}
\newacronym{cil}{CIL}{Carbamate Ionic Liquid}
\newacronym{tbp}{TBP}{\firstchar{T}ributyl phosphate}
\newacronym{kfecn}{K$_3$[Fe(CN)$_6$]}{\firstchar{P}otassium ferricyanide}
\newacronym{23dpg}{2,3-BPG}{2,3-BisPhosphoGlyceric acid}
\newacronym{ros}{ROS}{\firstchar{R}eactive oxygen species}
\newacronym{dabco}{DABCO}{1,4-diazabicyclo[2.2.2]octane}
\newacronym{teda}{TEDA}{\firstchar{T}riethylenediamine}
\newacronym{mof}{MOF}{Metal Organic Framework}

% Measurement technique
\newacronym{ppg}{PPG}{\firstchar{P}hotoplethysmography}
\newacronym{nirs}{NIRS}{Near Infra-Red Spectrometry}
\newacronym{ndir}{NDIR}{Non Dispersive Infra-Red}
\newacronym{mri-er}{MRI}{Magnetic Resonance Imager}
\newacronym{mri-ing}{MRI}{Magnetic Resonance Imaging}
\newacronym{dlr}{DLR}{Dual Lifetime Referencing}
\newacronym{fdlr}{f-DLR}{frequency-based Dual Lifetime Referencing}
\newacronym{tdlr}{t-DLR}{time-based Dual Lifetime Referencing}
\newacronym{dfdlr}{DF-DLR}{Dual Frequency Dual Lifetime Referencing}
\newacronym{drs}{DRS}{Diffuse Reflectance Spectometry}
\newacronym{fret}{FRET}{Fluorescent Resonance Energy Transfer}
\newacronym{ife}{IFE}{Inner Filter Effect}

% Mathematical methods
\newacronym{pca}{PCA}{Principal Component Analysis}
\newacronym{pls}{PLS}{Partial Least Squares}
\newacronym{pdf}{PDF}{Probability Density Function}
\newacronym{anova}{ANOVA}{ANalysis Of VAriance}
\newacronym{manova}{MANOVA}{Multivariate ANalysis Of VAriance}
\newacronym{dft}{DFT}{Discrete Fourier Transform}
\newacronym{rmse}{RMSE}{Root Mean Square Error}
\newacronym{snr}{SNR}{Signal to Noise Ratio}
\newacronym{snrdb}{SNR$_\text{dB}$}{Signal to Noise Ratio in dB}
\newacronym{sd}{SD}{Standard Deviation}
\newacronym{lotus}{LOTUS}{Law Of The Unconscious Statistician}
\newacronym{crlb}{CRLB}{Cram\'er-Rao Lower Bound}
\newacronym{lclt}{L-CLT}{Lyapunov's Central Limit Theorem}

% Physics
\newacronym{ir}{IR}{Infra-Red}
\newacronym{nir}{NIR}{Near Infra-Red}
\newacronym{stp}{STP}{Standard Temperature and Pressure}
\newacronym{nh}{NH}{Non-Heated}

% Shortcuts
\newcommand{\bscy}{Biosency}
\newcommand{\ssel}{Stow-Severinghaus electrode}
\newcommand{\aka}{\textit{a.k.a.}}
\newcommand{\conj}[1]{\overline{\mbox{$#1$\mathstrut}}} % math conjugate
\DeclareMathSymbol{\cdot}{\mathord}{symbols}{"01} % shortens space around cdot

% Latin locution
\newcommand{\etal}{\textit{et al.}}
\newcommand{\ie}{\textit{i.e.}}
\newcommand{\eg}{\textit{e.g.}}
\newcommand{\etc}{\textit{etc}}
\newcommand{\perse}{\textit{per se}}
\newcommand{\invivo}{\textit{in vivo}}
\newcommand{\invitro}{\textit{in vitro}}

% Abreviation: label, name, description
\newglossaryentry{pao2}{name={paO$_2$},description={Arterial dioxygen partial pressure}}
\newglossaryentry{po2}{name={pO$_2$},description={Dioxygen partial pressure}}
\newglossaryentry{pco2}{name={pCO$_2$},description={Carbon dioxide partial pressure}}
\newglossaryentry{paco2}{name={paCO$_2$},description={Arterial carbon dioxide partial pressure}}
\newglossaryentry{petco2}{name={petCO$_2$},description={End-tidal carbon dioxide partial pressure}}
\newglossaryentry{ptco2}{name={tcpCO$_2$},description={Transcutaneous carbon dioxide partial pressure}}
\newglossaryentry{pseco2}{name={p$_\text{Se}$CO$_2$},description={Inner sensor carbon dioxide partial pressure}}
\newglossaryentry{tcpo2}{name={tcpO$_2$},description={Transcutaneous dioxygen partial pressure}}
\newglossaryentry{pvaco2}{name={p$_{\text{v-a}}$CO$_2$},description={Carbon dioxide venous-to-arterial pressure difference}}
\newglossaryentry{h2o}{name={H$_2$O},description={Water}}
\newglossaryentry{h2s}{name={H$_2$S},description={Hydrogen sulfide}}
\newglossaryentry{sodith_for}{name=Na$_2$S$_2$O$_4$,description={Sodium dithionite}}
\newglossaryentry{sobisulf_for}{name=Na$_2$S$_2$O$_5$,description={Sodium metabisulfite}}
\makeglossaries
\glsdisablehyper

%%%%%%%%%%%%%%%%%%%%%%%%%%
% Thesis related commands
%%%%%%%%%%%%%%%%%%%%%%%%%%

% Default figure placement
\makeatletter
\renewcommand*{\fps@figure}{!htbp}
\makeatother

% Hidden link
\newcommand{\mhiddenlink}[2]{\begingroup\hypersetup{hidelinks}\href{#1}{#2}\endgroup}

% tcolobox for book citation
\newcommand*{\mbookcite}[4]{\begin{center}
		\begin{minipage}{0.8\linewidth}
			\enquote{#1}
			
			\raggedleft --- #2, \textit{#3} (#4)
		\end{minipage}
\end{center}}

% Further Research Is Needed stuff
\newcounter{mfrincounter}
\newcommand{\mfrin}{\begin{marginfigure}[0.2cm]\centering \includegraphics{0_front_matter/front_figures/tikz/out/frin.pdf}\end{marginfigure}\stepcounter{mfrincounter}}

\makeatletter
\newcommand\inlineparagraph
{%
	\@startsection{paragraph}{4}{\z@}{3.25ex \@plus 1ex \@minus .2ex}{-1ex}
	{\normalfont\normalsize\bfseries}
}
\makeatother

% tcolobox for TL;DR sections
\newtcolorbox{tldrbox}{
	enhanced,
	coltitle=black,
	colbacktitle=white,
	colback=white,
	attach boxed title to top center=
	{yshift=-0.25mm-\tcboxedtitleheight/2,yshifttext=2mm-\tcboxedtitleheight/2},
	boxed title style={boxrule=0.5mm,
		frame code={ \path[tcb fill frame] ([xshift=-4mm]frame.west)
			-- (frame.north west) -- (frame.north east) -- ([xshift=4mm]frame.east)
			-- (frame.south east) -- (frame.south west) -- cycle; },
		interior code={ \path[tcb fill interior] ([xshift=-2mm]interior.west)
			-- (interior.north west) -- (interior.north east)
			-- ([xshift=2mm]interior.east) -- (interior.south east) -- (interior.south west)
			-- cycle;} },
	colframe=black,
	title=\acs{tldr}\vphantom{\rule{1mm}{3.8mm}},
	arc=0mm,
	outer arc=0pt,
	fonttitle={\large\bfseries},
	segmentation engine=path,
	segmentation style={draw=black, line width=1.5pt, solid},
	after skip=20pt,}

\newtcolorbox{appbox}{
	enhanced,
	coltitle=black,
	colbacktitle=white,
	colback=white,
	colframe=black,
	arc=0mm,
	outer arc=0pt,
	fonttitle={\large\bfseries},
	segmentation engine=path,
	segmentation style={draw=black, line width=1.5pt, solid},
	before skip=20pt,
	after skip=20pt,
	}

% tcolobox for important stuff
\newtcolorbox{keypointbox}{center, arc=0mm, outer arc=0pt,colback=white, width=0.85\linewidth}

\setcounter{secnumdepth}{4}