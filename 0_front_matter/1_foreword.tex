\chapter{Foreword}\label{forechap:foreword}

\begin{tldrbox}
	Each chapter of this PhD thesis has a \gls{tldr} header that summarises the key points of the said chapter. You are \textbf{strongly encouraged} to read those, especially when skim-reading this document.  You may also use the clickable links below to quickly navigate between chapters. Have a nice journey.
	
	\tcblower
	
	 \hphantom{Previous chapter} \hfill \hyperref[chapter:toc]{Main Table Of Content (TOC)} \hfill \hyperref[forechap:ack]{Next chapter}
	
\end{tldrbox}

\mbookcite{The most merciful thing in the world, I think, is the inability of the human mind to correlate all its contents. We live on a placid island of ignorance in the midst of black seas of infinity, and it was not meant that we should voyage far. The sciences, each straining in its own direction, have hitherto harmed us little; but some day the piecing together of dissociated knowledge will open up such terrifying vistas of reality, and of our frightful position therein, that we shall either go mad from the revelation or flee from the deadly light into the peace and safety of a new dark age.}{Howard Phillips Lovecraft}{The Call of Cthulhu}{1928}

Despite Lovecraft's fair warning, humanity has always endeavoured to push back the frontiers of its knowledge, further lifting the veil that used to cover the unfathomable abysses of ignorance whence it emerged. Just as the navigators of yore once outlined the contours of new, strange, and beautiful worlds, today's scientists are tirelessly exploring and meticulously mapping out the seemingly infinite expanse of scientific knowledge. As one of them, I took great pleasure in venturing into the different fields of research that my works required me to explore, and in trying to build bridges between the knowledge of old, and tomorrow's findings. Fortunately, while doing so, I managed to evade the crawling madness which often seals the destiny of Lovecraft's protagonists.

It may be seen from the many citations that sprinkle this thesis that I very much enjoy reading papers, thereby making mine the wisdom of elders. This foot in the past allowed me---at least hopefully to some extent---to derive a comprehensive view of the problem at hand, and it is my fervent wish to provide the reader with an overview of the ins and outs of the issues addressed. As for the future, in addition to the several novelties that this thesis brings to light, I also strived to clearly indicate the limitations of the presented work, as well as the future research avenues that could be explored.

\mfrin{}When this happens, the \enquote{further research is needed} magnifying glass icon that can be seen on the right hand side of the present text will be used. As you shall see, I tried to use the many possibilities offered by \LaTeX{} in order to offer what I intend to be an enjoyable reading experience. Of note, this thesis is \textbf{not} meant to be printed on paper: not only for ecological reasons, but also due to the many clickable links that strew it. \href{https://archive.org/details/lshort}{Turquoise links} are web links, green links\cite{oetiker2007} concern citations that will lead you to the corresponding bibliography entries, and \hyperlink{forechap:foreword}{red links} are for other numbered elements, such as chapters, equations, figures, tables, \emph{etc.} Do not fear clicking on a citation link, as citations are back-referenced, meaning that there is also a link in the reference list to jump back to whence you came from, thanks to \texttt{biblatex}'s package \texttt{backref} option.
\clearpage
A great deal of efforts have been put into being as educational as possible, especially because of the intrinsic multi-disciplinarity of the presented works. It is not expected from the reader to be an expert in all the covered fields\footnote{Although being initiated with some of the presented arcana, I do not pretend being one myself. It would be presumptuous after only five years working on multiple topics at the same time to compare myself with scholars who devoted their lives to a single one.}, and a brief context is often provided, along with a reasonable amount of literature references to dig into, if need be. If I truly hope that this educational nature---as well as the extended discussions proposed---make this thesis more than a proof-of-work in the academic system, \ie{} by making it a useful resource for those who may come after me, I am well-aware of the lengthy result that ensues. Consequently, I have included a \gls{tldr} text box at the beginning of each chapter to provide the reader with the pith and marrow of the said chapter's content.

Finally, I hope that you will enjoy reading this thesis as much as I enjoyed conducting the corresponding research and writing the present manuscript. A GitHub repository containing all the source files is available \href{https://github.com/e-dervieux/phd_thesis}{here} (including raw data and TikZ / PGFPlots plotting codes). Commit \hl{XXX} corresponds to the final version, as validated by the doctoral committee whose members are given in the \hyperlink{frontpage}{front page}. If you spot any misinterpretation, typo, or error of any sort, please report it by writing me  \href{mailto:emmanuel.dervieux@gmail.com}{here}, and always consult the afore-mentioned repository for the latest version available.

\vfill

\hl{Reste à faire:}
\begin{enumerate}
	\item[--] homogénéiser notations $K_a$, K$_a$, K$_\text{a}$
%	\item[--] faire une passe sur DOIs stockés dans sheets
%	\item[--] regarder Zijlstra2000 trouver page mention turbidité
%	\item[--] debug: textwidth in cm: \printinunitsof{cm}\prntlen{\textwidth}
	\item[--] obtenir les permissions pour Hodgkinson et Barrington
	\item[--] corriger les dépassements de marge et vérifier la mise en page générale (à faire après les retours des rapporteurs)
%	\item[--] faire un code couleur de surlignage, type \hl{jaune} pour "à modifier" et \hl[lightgreen]{vert} pour "corrigé"
%	\item[--] regarder draxler1995 pour perméabilité PS vs PAN
	% s'assurer que la ref 666 pointe vers un truc ésotérique
	% TODO: il reste un todo caché en intro pour changer les chiffres romains en cerclés
\end{enumerate}

\vfill
\setcounter{secnumdepth}{3}

\paragraph{Tools that were used to write this thesis} \hphantom{text}

This thesis was written using \LaTeX{}, for which I highly recommend reading \emph{The Not So Short Introduction to \LaTeX}, by Tobi Oetiker\cite{oetiker2007}. Data plots were mostly composed using TikZ / PGFPlots, and Python programmers may find the \href{https://web.archive.org/web/20240506065201/https://pypi.org/project/tikzplotlib/}{\texttt{tikzplotlib}} library particularly helpful for transitioning from Matplotlib to Tikz. Schematics were edited using Inkscape, with Gimp used for preprocessing raster images. WebPlotDigitizer was also seldom used to retrieve raw data from old figures. Most vector symbols are sourced from \href{https://web.archive.org/web/20240505215016/https://thenounproject.com/}{The Noun Project}, which offers a comprehensive collection of icons with permissive licensing (CC BY 3.0, corresponding acknowledgments are provided in \hyperref[chap:add_ack]{back matter}). 3D designs were modelled using Onshape, a web-based, free, Solidworks-like software, while \gls{pcb} designs were achieved thanks to KiCad. Python was predominantly used for programming, data analysis, and statistical hypothesis testing, with occasional use of R for specific statistical explorations. Of note, while ChatGPT 3.5 / 4-o / 4-mini was used when redacting this manuscript, its use was limited to translation or re-wording purposes, \textbf{not} for redacting entire paragraphs / sections.

\setcounter{secnumdepth}{4}