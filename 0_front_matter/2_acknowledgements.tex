\chapter{Acknowledgements}\label{forechap:ack}

\begin{tldrbox}
	\centering
	\mhiddenlink{http://web.archive.org/web/20240321021520/https://en.wikipedia.org/wiki/Evangelion:_1.0_You_Are_(Not)_Alone}{I was (not) alone.}
	
	\tcblower
	
	\hyperref[forechap:foreword]{Previous chapter} \hfill \hyperref[chapter:toc]{Main Table Of Content (TOC)} \hfill \hyperref[chap:intro]{Next chapter}
	
\end{tldrbox}

This thesis is the result of a tripartite collaboration spanning a thousand kilometers, between Biosency (Rennes), the Orphy laboratory (Brest), and the ICube laboratory (Strasbourg), which would never have been possible without the active support of numerous individuals who deserve recognition.

First and foremost, I would like to express my sincere gratitude to Marie Pirotais, Yann Le Guillou, and Quentin Bodinier, who trusted me when I knocked at Biosency's door in late 2018, looking for a subject for my doctoral thesis. They placed their full confidence in me, granting me carte blanche and complete freedom to carry out my research, for which I am extremely grateful. I also extend my thanks to the entire Biosency team, many of whom are both friends and colleagues, each contributing to the hard-working yet convivial atmosphere of this young start-up.

Second, I would like to warmly thank Michaël Théron for welcoming me during my weeks of experimentation in Brest; not only in his laboratory, but also in his home, where I readily felt as part of his family. I also want to express my thanks to Marie-Agnès Girous-Metgès and François Guerrero for their dynamic contributions to my works, as well as for their redacting advices. More generally, I cannot emphasise enough the very positive atmosphere that prevails in the Orphy laboratory, and I warmly thank its members both for their friendly welcome (and cookies), and for their participation as guinea pigs in the rate-based study detailed in Chapter~\ref{chap:tcco2}.

Third, I am grateful to Wilfried Uhring who---in addition to supervising my work and giving me lots of valuable advice---gave me a foothold in teaching, by entrusting me to design a lecture course on computer architecture from scratch. I am also particularly thankful to the people at ICube who strove to make my experiments feasible despite an erratic scheduling on my side and unending asbestos removal works. Of note, I want to express my warmest feelings for my fellow travellers in ICube Cronenbourg's laboratory---Lilas and Lucas---for their politically incorrect, saucy humour. They provided a much-needed respite and prevented me from losing what remained of my sanity while redacting this manuscript.

My profound thanks also go to the many teachers and scholars that gave me a strong taste for science and engineering throughout my school years. Foremost among them are my parents, who---as biology and physics teachers---instilled in me a passion for understanding and trying to explain the laws of the physical world. Their contribution is obviously not limited to that, and their ongoing support has been a great help to me throughout my whole (scientific) life.

Finally, I would like to say a few words of thanks to my friends and family, whose fellowship and affection always pushed me forward, and encouraged me to embark on numerous projects, both personal and professional. In particular, I would like to salute Joffrey for our \href{https://www.youtube.com/watch?v=zowdYJ5vWB0}{everlasting DIY projects}, and Raphaël for the \href{https://esoteroots.bandcamp.com/}{already released} and forthcoming music.

Of course, the support I received during my doctoral years was not limited to the few people mentioned above. I can readily think of dozens of names and faces who contributed to this work one way or another. Even if I cannot name each and every one of them here, I acknowledge their work, and deeply thank them for their contribution, however small.
















